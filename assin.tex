\documentclass[12pt, letterpaper]{article}
\usepackage[utf8]{inputenc}

\title{Biomedical Enginerring , Assingment 6}
\author{Somya Kabra }
\date{March 2022}
\begin{document}
\maketitle
\begin{center}
\huge{5 Solutions to Covid19 provided by Biomedical Engineers}
\end{center}
\rule{\textwidth}{0.5pt}
\subsection{By making testing devices for rapid and accurate COVID-19 diagnosis.}
\subsection{Internet of Things applications in COVID-19 hospitals.}


\subsection{Data Science and statistical modeling applied to COVID-19 tracking.}


\subsection{To build ventilators and PPE(personal protection equipment).}


\subsection{Use of 3D printering to produce additional face masks and visors for healthcare workers}
\rule{\textwidth}{0.5pt}
\maketitle
\begin{center}
\section{By Making Testing devices for rapid and accurate COVID-19 diagnosis}
\rule{\textwidth}{0.5pt}
Rapid and accurate antibody tests could play an important role as governments, medical workers, scientists and private citizens alike continue to navigate the pandemic. Antibody tests can reveal who has already been exposed to the virus and developed immunity, at least temporarily, and can safely go back to work. If done in a widespread way, they could show the true scale of the pandemic and its death rate. 

And the U-M researchers say their particular approach could give doctors critical, near-real-time insights into how a patient is responding to treatment, or a vaccine once one is developed.

Small-scale antibody testing has been done in some countries. Research projects are underway in the U.S. and while kits are beginning to materialize on the market here, they’re not yet widely available.

Antibody, or “serology” tests are different from the “PCR” tests being used to diagnose COVID-19. Rather than screen for the virus itself, serology tests detect antibodies—proteins the immune system manufactures to fight it. 
Biomedical Engineering Tools for Management of Patients with COVID-19 presents biomedical engineering tools under research (and in development) that can be used for the management of COVID-19 patients, along with BME tools in the global environment that curtail and prevent the spread of the virus. BME tools covered in the book include new disinfectants and sterilization equipment, testing devices for rapid and accurate COVID-19 diagnosis, Internet of Things applications in COVID-19 hospitals, analytics, Data Science and statistical modeling applied to COVID-19 tracking, Smart City instruments and applications, and more. Later sections discuss smart tools in telemedicine and e-health.
\rule{\textwidth}{0.5pt}
\section{Internet of Things applications in COVID-19 hospitals.}
During the COVID-19 pandemic, high-tech solutions like the Internet of Things (IoT) for smart buildings have been critical in keeping our urban societies functional . The aim of highlighting the smart building with IoT technologies is to identify COVID-19 cases, decrease the spread, and reduce the impact of the pandemic. Smart buildings, supported by IoT, were primarily relied upon for security, automated management and control, increasing energy efficiency, safety, usability, and accessibility. Furthermore, as lockdown eases, they will also help manage building occupancy levels and social distancing .

This research is aimed to improve smart buildings’ features by adding queue management, smart navigation of places to minimize people connection, and social distancing safety mechanisms. Besides, environmental advancement features to increase people’s safety are features that this project can offer in the subsequent development .

This solution implements through several steps can be seen in Figure 1. Our model-building to develop our proposed solution is a hospital, which is the most demanded place to make it smart based on surveys . As the first step, sensors’ positioning is deployed using our Building Information Modeling (BIM) and mathematical formulation. Next, the data will send to the gateway through several specific protocols. Then, received data will be stored in the servers and processed through the designed middleware; consequently, the processed data sent to the mobile application, shown in Figure 2 . The presented platform’s primary thinking is to propose an application that users can control through their smartphones using the data gathered through sensors and beacons located in the different parts of a hospital . IoT would be the best solution to combine all collected information and send it to the end-users 
\rule{\textwidth}{0.5pt}
\section{Data Science and statistical modeling applied to COVID-19 tracking.}
Data Science for COVID-19 presents leading-edge research on data science techniques for the detection, mitigation, treatment and elimination of COVID-19. Sections provide an introduction to data science for COVID-19 research, considering past and future pandemics, as well as related Coronavirus variations. Other chapters cover a wide range of Data Science applications concerning COVID-19 research, including Image Analysis and Data Processing, Geoprocessing and tracking, Predictive Systems, Design Cognition, mobile technology, and telemedicine solutions.
Provides a leading-edge survey of Data Science techniques and methods for research, mitigation and treatment of the COVID-19 virus.Integrates various Data Science techniques to provide a resource for COVID-19 researchers and clinicians around the world, including both positive and negative research findings.Provides insights into innovative data-oriented modeling and predictive techniques from COVID-19 researchers.Includes real-world feedback and user experiences from physicians and medical staff from around the world on the effectiveness of applied Data Science solutions
\rule{\textwidth}{0.5pt}
\section{To build ventilators and PPE(personal protection equipment.}
As the coronavirus disease 2019 (COVID-19) pandemic accelerates, global health care systems have become overwhelmed with potentially infectious patients seeking testing and care. Preventing spread of infection to and from health care workers (HCWs) and patients relies on effective use of personal protective equipment (PPE)—gloves, face masks, air-purifying respirators, goggles, face shields, respirators, and gowns. A critical shortage of all of these is projected to develop or has already developed in areas of high demand. PPE, formerly ubiquitous and disposable in the hospital environment, is now a scarce and precious commodity in many locations when it is needed most to care for highly infectious patients. An increase in PPE supply in response to this new demand will require a large increase in PPE manufacturing, a process that will take time many health care systems do not have, given the rapid increase in ill COVID-19 patients.
In its current guidance to optimize use of face masks during the pandemic, the Centers for Disease Control and Prevention (CDC) identifies 3 levels of operational status: conventional, contingency, and crisis.1 During normal times, face masks are used in conventional ways to protect HCWs from splashes and sprays. When health care systems become stressed and enter the contingency mode, CDC recommends conserving resources by selectively canceling nonemergency procedures, deferring nonurgent outpatient encounters that might require face masks, removing face masks from public areas, and using face masks for extended periods if feasible.
When health systems enter crisis mode, the CDC recommends cancellation of all elective and nonurgent procedures and outpatient appointments for which face masks are typically used, use of face masks beyond the manufacturer-designated shelf life during patient care activities, limited reuse, and prioritization of use for activities or procedures in which splashes, sprays, or aerosolization are likely. 
\rule{\textwidth}{0.5pt}
\section{Use of 3D printering to produce additional facemasks and visors for healthcare workers.}
The COVID-19 outbreak was first reported in Wuhan, China in December 2019, resulting in a worldwide public health threat . The race to obtain medical supplies reflects a global panic over a dwindling supply of N95 respirator masks, face shields, ventilators, testing kits and other personal protective equipment (PPE). Adequate production of PPE is essential during the COVID-19 pandemic to protect healthcare workers from viral transmission. 3D printing can be used to create intricate architectures to aid with these shortages. 3D printing is an integrated approach to robotic fabrication, using computer-aided design (CAD) systems to deposit layers of biomaterials (within external anatomy, within internal anatomy and replacement parts for devices) [5–10]. The success of a medical device is not only dependent on the type of biomaterial used for its fabrication but also on the structural integrity and quality (defect free) of the printing parts. Additive manufacturing technologies have opened new opportunities for manufacturing and production paradigms . The primary advantage of using additive manufacturing is for on-demand and redistributed manufacturing to circumvent the supply chain disruption. Moreover, additive manufacturing allows for lower energy costs, reduced waste and is affordable. Ideal biomaterials should be readily printable, mechanically stable and biocompatible . With ongoing materials research used in 3D technology, there is potential for innovative and cost-effective applications for addressing this current global crisis. Furthermore, the primary advantage of using additive manufacturing is for on-demand and redistributed manufacturing to circumvent the supply chain disruption. This review summarizes the key elements and advantages of 3D technologies that can be used to create 3D-printed tools to protect healthcare workers during the COVID-19 pandemic.

3D-printing techniques
Extrusion-based printing
Extrusion-based printing utilizes print nozzles that extrude material by air pressure or mechanical force, with continuous printing in a layer-by-layer design for controlled and accurate deposition. Synthetic polymers that are commonly used in extrusion printing include, acrylonitrile butadiene styrene (ABS), polyurethane polyvinylpyrrolidone, polyvinyl alcohol and polylactic acid.

The most common type of extrusion-based printing utilized is fused-deposition modeling (FDM). FDM is fast, effective, and allows for easy integration with different CAD softwares. FDM uses thermoplastic filaments that pass through multiple heated printer nozzles and can therefore print multiple types of materials simultaneously. Nylon, ABS, polylactic acid, polyvinyl alcohol, polycarbonate (PC) and polycaprolactone can be printed by FDM . Furthermore, FDM can be utilized to build constructs in a timely manner with 3D accuracy and excellent mechanical properties. Thus, FDM can be used to create customized patient and physician-specific medical devices, such as masks, face shields, ventilator valves that can be used during the COVID-19 pandemic.
\rule{\textwidth}{0.5pt}
\end{center}
\end{document}